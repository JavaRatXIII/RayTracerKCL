\documentclass{article}
\usepackage[utf8]{inputenc}
\usepackage{graphicx}
\usepackage{biblatex}


\title{Ray Tracer}
\author{MiddleEast Group }
\date{March 2018}

\begin{document}
\maketitle 

\section{Introduction}

Ray Tracing is a rendering technique which takes simple 2D objects such as circles, squares or rectangles and
produces 3D rendering of these objects in the form of spheres, cubes, cuboids etc.  by tracing the path of light as pixels on a screen space and imitating the effects of the light  with 3D objects. Scenes in ray tracer are defined mathematically by the developer. In addition, it can take data from images and models captured by means such as digital photography. \newline An advantage of ray tracing is that it gives a more realistic and desirable method of rendering. Many physically correct experience can be demonstrated with ease using a ray tracing algorithm, since the algorithm simulates the sign of light in the real world.
However, it is extremely time-consuming to find the intersection between rays and geometry (scratchapixel.com). The main issue with ray tracing is the speed. Anyhow, as computers become more quicker, it is less and less of an obstacle. 

\section{Review}
\subsection{What is Ray Tracer?}

\paragraph{}

To understand ray tracer you need to know what is "ray"? Basically,a ray is a line that starts off at a point in space towards some direction. When a ray hits a surface it is either reflected or refracted.
In the real world, light is reflected on object surfaces until it noticed by the observer (camera or person). \newline Now, this method can not be used while using a computer to simulate a ray tracer. Hence, a reverse mechanism is used where the rays are sent out from the observer viewpoint. This is much better in terms of computational speed. 
\paragraph{}
Ray tracer technique is based on an algorithm where it builds an image pixel by pixel. It casts additional rays from the hit point to determine the pixel color as shown in Figure 1.
Once the nearest object has been identified, the algorithm will estimate the incoming light at the point of intersection, examine the material properties of the object, and combine this information to calculate the final color of the pixel (blog.world-mysteries.com). In other words we can say that if the light hits the surface and not be blocked,  The shading of the surface is computed using traditional 3D computer graphics shading models. 

\begin{figure}
    \centering
    \includegraphics{F1.png}
    \caption{Basic ray tracing}
    \label{figure1}
    \soucre{
http://i.imgur.com/YPSKk.png}
\end{figure}

\subsection{Objective}

\paragraph{}
This project aims at developing a Ray Tracer software which consists of two parts. The first part is the GUI, that will allow the end users to insert preferable values in the form of coordinates (x, y) which will be sent to the back-end system as a text file.
The second part, which is the back-end system will take the input and produce a 3D image for the end user and this image will be saved as PNG file. 
The software will be based on C++ language since all the team members are confident with this language and it is proved that it is fast compared to other programming languages. It has a compiling time of 1.5 seconds according to \textit{ffconsultancy.com }.


\section{Design}

\subsection{Requirements}


To achieve our objectives, we need to deliver a ray tracer software (User interface and Back-end system). 
Basically, our ray tracer software depends on two parts, Front end (GUI) and back-end system. The front end provides a several selection for users where he/she can choose the shape, size and colors. The front-end system will save user preferences as a text file and send it to the back-end system. 
The back-end system will run and implement the shapes based on user choices.



\subsection{Design}

Our ray tracer software based on object-oriented design. We have decided to write our program using C++ since it enables the developers to write complicated programs and simplify them into parts called functions. 
Benefits of C++
\begin{itemize}
    \item 
	C++ is a highly portable language and is often the language of choice for multi-device, multi-platform app development.
\item	C++ is an object-oriented programming language and includes classes, inheritance, polymorphism, data abstraction and encapsulation.
\item	C++ has a rich function library.
\item	C++ allows exception handling, and function overloading which are not possible in C.
\item	C++ is a powerful, efficient and fast language. It finds a wide range of applications – from GUI applications to 3D graphics for games to real-time mathematical simulations. [1]

\end{itemize}

\section{Implementation}
\subsection{Implementation}
To start working with ray tracer software we have to learn the basic things of ray tracer concept and it structure. Many of sites and tutorials helped us to build our knowledge and to have better understanding. The front-end system was built through QT creator software where it depends on several classes. Issues: save as text file? How did we resolved it 
 
During the implementation we straggled on reading user’s input that we received as a text file from the front-end and we were able to manage it by converting from abstract type string into concrete type string which has solved the problem as you can see in below code: 

The back-end system was build through CodeBlocks software, in help with learning tutorial [2]. Basically, receive a text file from the front-end and it will read it as variables. These variables will be converted to data values that can be used to draw the shape in terms of the size and other aspects of the shape.  


Below code shows how to read the data from text file that was pushed from the front-end. This was in to enable us to manipulate the data to be able to draw shapes.

A problem occurred when trying to run the code in different machines, as the user name will changed according to the user’s machines name causing the directory used in the code to be incorrect. To solve this problem we used DWORD type, as it will take the user name’s length from the system. In addition, we used GetUserName method where it use character array and the user name’s length to retrieve the user’s machine name. 


\begin{lstlisting}
string output;
       string shape = "";
       string scale = "";
       double shapeSize = 0.5;
       string color = "";
       string translucent = "";
       string reflection = "";

       char uname[UNLEN+1];
       DWORD username_len = UNLEN+1;
       GetUserName(uname, &username_len);
       string username(uname);

       std::stringstream ss;
       //ss << "C:\\Users\\" << username << "\\Documents\\A Kings\\Group Work\\RayTraceFrontEnd\\Scene.txt";
       ss << "C:\\Users\\" << username << "\\Documents\\GitHub\\RayTracerKCL\\RayTraceFrontEnd\\Scene.txt";
       std::string fileLocation = ss.str();

       ifstream myReadFile;
       myReadFile.open(fileLocation);

       int i = 0;

       if (myReadFile.is_open())
       {
           while (!myReadFile.eof())
           {
               for(i; i < 5; i++)
               {
                   getline(myReadFile,output);
                   if(i == 0)
                   {
                       shape = output;
                       break;
                   }
                   else if (i == 1)
                   {
                       scale = output;
                       break;
                   }
                   else if(i == 2)
                   {
                       color = output;
                       break;
                   }
                   else if(i == 3)
                   {
                       translucent = output;
                       break;
                   }
                   else
                   {
                       reflection = output;
                       break;
                   }
               }
           i++;
           }
       }



\end{lstlisting}

Another section in the code, where it implement all the intersections and makes the object visible in the scene, so it will not show as a black scene.  It does the light intersections with the object so that they have shading. It also has the second hand intersection where it enables surfaces to be reflective.  
\begin{lstlisting}

Color intersection_color = getColorAt(intersection_position,
intersecting_ray_direction, scene_objects,
index_of_winning_object,
light_sources, accuracy, ambientlight); \end{lstlisting}

One important variable is (aadepth) where it does the anti-aliasing function making the images smoother. In that way we do not have edges on a smooth surface.  Basically, this variable will start from value (1) and above. But the higher the value, the more time it will take to render the image and effecting the processing time. And although, the image will be nicer and smoother, it will take too long to render the image as it will take over 10 seconds to create the image.  Therefore, we settled to maintain the value at 1.
\begin{lstlisting}
int aadepth = 1;
\end{lstlisting}

Our code depends on saving PNG images, to do this we had to use below method to convert PPM into PNG, as there no standard libraries to implement this. 
\begin{lstlisting}
savepng("scene.png",width,height,dpi,pixels);
\end{lstlisting}


The Object Class is the parent class to all object classes, so no objects will be possible to be in the scene without this class. As the Sphere and Plane classes are children of the Object class and the Cube and Pyramid are implemented using the Triangle class, which too is a child class of the Object class.

\begin{lstlisting}
class Object {
    public: 

    Object (); 

    
    virtual Color getColor () { return Color (0.0, 0.0, 0.0, 0); } 
    virtual Vect getNormalAt(Vect intersection_position) {
        return Vect (0, 0, 0);
    }

   // function that returns ray intersection
    virtual double findIntersection(Ray ray) {
        return 0; // for no objects intersection is 0
    }

};

\end{lstlisting}
 \newpage
\subsection{Testing}
Any project need to be tested in different ways to make sure that it meets the project’s objectives. For that you demonstrated several case scenarios to check the performance and the result of the program. \newline

\begin{itemize}
    \item First case scenario: \newline
    User will choose Square shape, scale 25 percent, in pink color. There will be no transparency, but there will be reflectivity. 
    
   \begin{figure}[H]
\centering
\includegraphics[scale=1]{T1.png}
\caption{The Unit Circle \label{fig1}}
\end{figure}

 
 \begin{figure}
    \centering
    \includegraphics{T1-1.png}
    \caption{Result}
    \label{fig3}
\end{figure}



\item Second case scenario: \newline
User will choose Circle shape, scale 50 percent, in Purple color. There will be a transparency and reflectivity. 

\begin{figure}[H]
\centering
\includegraphics[scale=.5]{T2.png}
\caption{The Unit Circle \label{fig1}}
\end{figure}

\begin{figure}
    \centering
    \includegraphics{T2-2.png}
    \caption{Result}
    \label{fig5}
\end{figure}

\item Third case scenario: \newline
User will choose Circle shape, scale 100 percent, in Purple color. There will be no transparency but there will be reflectivity.

\begin{figure}[H]
\centering
\includegraphics[scale=.5]{T3.png}
\caption{The Unit Circle \label{fig1}}
\end{figure}

\begin{figure}
    \centering
    \includegraphics{T3-3.png}
    \caption{Result}
    \label{fig5}
\end{figure}

\item Forth case scenario: \newline
User will choose Circle shape, scale 0 percent, in yellow color. There will be no transparency but there will be reflectivity.

\end{itemize}



 \newpage



\section{Team Work}
The team has decided to divide the technical work into two halves. Half of the team will work on GUI and the other half will handle the back-end system, as well as the report and the presentation. However,  the main roles and responsibilities were divided as below. Each member dedicated to achieving her tasks.
\begin{itemize}

\item \textbf{GUI and report}: Norah Alsuwily, Maryam Yasin, Sara Alotaibi. 
\item \textbf{Back-end and presentation}: Roshini Ashokkumar, Nandhini Sreekumar, Tamanna Qureshi. 
\end{itemize}
This way we are able to efficiently manage the load of this project to ensure that every team member is allocated with the balanced workload. Thus, ensuring that the project proceeds smoothly and successfully.



At the beginning of this project, the Middle-East team sat a plan to follow as you can see in below table. 

\begin{center}
\begin{tabular}{ |c|c|c|} 
\hline
\textbf{Week no.} & \textbf{Week Date } & \textbf{To-do } \\ 
\hline
Week 1 & 14 January  & Brainstorming \\ 
\hline
Week 2 & 21 January  & Discussion  on tools and language to be used. \\ 
\hline

Week 3 & 28 January & Search and share knowledge. 
 Also, starting with GUI and Back-end system. 
 \\ 
\hline
Week 4 & 4 February & Finalizing the initial report and presentation.  \\ 
\hline
Week 5 &  11 February & Working on the software (GUI and Back-end)  \\
\hline
Week 6 & 18  February & Working on the software (GUI and Back-end)   \\ 
\hline
Week 7 & 25  February & Working on the software (GUI and Back-end)   \\ 
\hline
Week 8 & 4 March  & Working on the software (GUI and Back-end) + Working on the final report.   \\ 
\hline
Week 9 & 11 March  & Testing and handling errors. + Working on the final report.   \\ 
\hline
Week 10 & 18 March  & Preliminary final report. \\ 
\hline
Week 11 & 25 March  & *Finalizing the project  \\ 

\hline

\end{tabular}
\end{center}

*(Check the system is running successfully and practice for the presentation) 
 


\subsection{Processes and Tools}

This section tells about the team's operations and tools used to manage the work as below.   
\begin{itemize}


\item Meeting and Communication: Team groups in every project need to communicate in a formal way to follow the time line and watch the progress for their project. As well as negotiate and get feedback to enhance and reflect team’s suggestions. To do that we used couple of applications like Slack and Whatsapp to be up to date. As well as physical meetings in the college, which was every Tuesday at 1 O’clock since it suites all team members to meet and discuss previous and next agenda. 
\item Development: Our system was divided into parts where we used to two softwares to develop the code. We chose QT creator and CodeBlock since they are:
\begin{itemize}
\item 	Easy to use.
\item	No need to use another program where you can compile within the same software.
\item Data visualization where 

\end{itemize}
\item Documentation: Each project needs to be documented and presented in a good way. There are many documentation programs, however, we chose Latex and Powerpoint since they satisfy college requirement, free and easy to use. 



\end{itemize}



\subsection{Meeting Management} 
The team has agreed to meet regularly every week on Tuesday and divide the tasks fairly according to each teammate's skills. The progress will be monitored every week and comments from every team member will be adhered to. \newline If there are any personal circumstances that will prevent the teammates from completing the task, she will need to follow it up in the following week and failure to do so, will definitely reflect on her marks based on Berger algorithm at the end of the project as the marks are based on her work and efforts that she has put into this project in this way we can efficiently handle peer assessment without any conflicts. 

\section{Evaluation}
\section{Peer Assessment}
As a team we have collaborated with each other to make sure that we accomplish our assignment successfully.  Each member has contributed with their effort and knowledge to ensure delivering the tasks in the right time, and without a single member of this team, this would not be completed. Based on member’s work we divided 100 points equivalently. The below table list team members alphabetically with their deserved points.  

\begin{center}
\begin{tabular}{ |c|c|} 
\hline
\textbf{Name} & \textbf{Points }  \\ 
\hline
Maryam Yasin  & 16.66 Points    \\ 
\hline
Nandhini Sreekumar  &  16.66 Points   \\ 
\hline

Norah Alsuwily  & 16.66 Points  
 \\ 
\hline
Sara Alotaibi & 16.66 Points  \\ 
\hline
Roshini Ashokkumar  &  16.66 Points \\ 
\hline
Tamanna Qureshi  & 16.66 Points  \\ 
\hline

\end{tabular}
\end{center}


\end{document}
